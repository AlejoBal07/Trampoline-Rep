\documentclass[11pt]{article}
\usepackage{geometry}                % See geometry.pdf to learn the layout options. There are lots.
\geometry{a4paper}                   % ... or a4paper or a5paper or ... 
%\geometry{landscape}                % Activate for for rotated page geometry
%\usepackage[parfill]{parskip}    % Activate to begin paragraphs with an empty line rather than an indent
\usepackage{graphicx}
\usepackage{amssymb}
\usepackage{epstopdf}
\DeclareGraphicsRule{.tif}{png}{.png}{`convert #1 `dirname #1`/`basename #1 .tif`.png}

\usepackage{hyperref}
\usepackage{underscore}
\usepackage{textcomp}
\usepackage{xcolor}
\usepackage{parskip}

\newcommand{\normaltilde}{{\raise.17ex\hbox{$\scriptstyle\mathtt{\sim}$}}}
\newcommand{\unixcl}[1]{\texttt{\fcolorbox{black}{gray!20}{#1}}}
\title{Trampoline Install}
\author{}
%\date{}                                           % Activate to display a given date or no date

\begin{document}
\maketitle

Users are not allowed to do system-wide installations of software on Linux computers of ECN. So before using Trampoline, you have to install it in your home directory.

\section{Trampoline}

\begin{enumerate}
\item Go in your home directory if you are not already in it: \unixcl{cd}
\item Download the package from here: \url{http://www.irccyn.ec-nantes.fr/~bechenne/trampoline}
\item Uncompress it in your home: \unixcl{tar xvzf trampoline.tgz}
\item Go in the viper\footnote{Viper is a program which allows to emulate interrupts on Unix and which is needed by Trampoline} directory: \unixcl{cd trampoline/viper}
\item Build viper: \unixcl{make}
\end{enumerate}

\section{Goil and miscellaneous}

Goil is the OIL compiler.

\begin{enumerate}
\item Go back in your home: \unixcl{cd}
\item Create a bin directory in your home: \unixcl{mkdir bin}
\item Go in the goilv2 makefile directory: \unixcl{cd trampoline/goilv2/makefile-unix}
\item Build it: \unixcl{make goil}
\item Put it in bin: \unixcl{mv goil \normaltilde/bin}
\item Edit your .bashrc in your home\footnote{On another Unix platform, it could be .profile} or create it if it does not exist and add the following lines at the end: \unixcl{export PATH=\$PATH:\normaltilde/bin}. Add the path to Viper too: \unixcl{export VIPER_PATH=\normaltilde/trampoline/viper}. And a last the path of the templates for goil: \unixcl{export GOIL_TEMPLATES=\normaltilde/trampoline/goilv2/templates}.
\item Update your environment: \unixcl{source \normaltilde/.bashrc}
\item Test goil is working: \unixcl{goil --version} should print \unixcl{./goil : version 2.1b2 (goilv2)}
\end{enumerate}

\section{Check everything works}

Trampoline includes a test suite which can be run to check everything works.

\begin{enumerate}
\item Go to the check directory which is located in the trampoline directory:\\
\unixcl{cd \normaltilde/trampoline/checkv2}
\item Run the test suite by typing \unixcl{./tests.sh}. Each test prints its name. 
. Then, after a while, \unixcl{Functional tests Succeed} should appear.
\end{enumerate}
%\section{}
%\subsection{}



\end{document}  