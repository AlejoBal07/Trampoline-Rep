
%\newpage
\chapter{Appendix}
%%%%%%%%%%%%%%%%%%%%%%%%%
\section{Launching Trampoline tests} \label{tests}
To launch the tests you have to compile ViPER \footnote{Virtual Processor Emulator, ViPER is used on Posix system to send interrupts to Trampoline to emulate the timers. It is launched by Trampoline.} first. Go in viper/ and type in a terminal :
	\begin{verbatim}
	$make
	\end{verbatim}
To launch the tests, go in check/ and type in a terminal :
	\begin{verbatim}
	$./tests.sh
	\end{verbatim}
At the end of the tests you should see :
	\begin{verbatim}
	...
	Compare results with the expected ones...
	Functional tests Succeed!!
	GOIL tests Succeed!!
	\end{verbatim}
If an error occurs, you can visit Trampoline's forum (\href{http://trampoline.rts-software.org/bb/}{http://trampoline.rts-software.org/bb/}).


%%%%%%%%%%%%%%%%%%%%%%%%%
\section{Cross-Compile an application} \label{compileanapplication}
To cross-compile a Trampoline application for ARM for example, you need to set  :
\begin{verbatim}
COMPILER = "arm-elf-gcc";
ASSEMBLER = "arm-elf-as";
LINKER = "arm-elf-ld";
\end{verbatim}
in your oil file as you can see in examples/arm/nxt/nxt\_simple.oil.\\
For the Lego Mindstorm NXT2.0 only, you also need to add the path to your libgcc and libc as below (X.X.X is your cross-gcc version) :
\begin{verbatim}
LDFLAGS = "-L[GNUARM_PATH]/lib/gcc/arm-elf/X.X.X -lgcc";
LDFLAGS = "-L[GNUARM_PATH]/arm-elf/lib -lc";
\end{verbatim}

And then, compile your application typing in a terminal (from the example in examples/arm/nxt/simple) :
	\begin{verbatim}
	$goil -t=arm/nxt --templates=../../../../goil/templates -g -i nxt_simple.oil
	\end{verbatim}
This will generate the Makefile needed, thus type :
	\begin{verbatim}
	$make
	\end{verbatim}
Then you need to upload the nxt\_simple\_exe.rxe file after installing drivers and softwares on your platform.

