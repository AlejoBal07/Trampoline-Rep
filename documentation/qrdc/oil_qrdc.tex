\documentclass[10pt,notumble]{leaflet}   	% use "amsart" instead of "article" for AMSLaTeX format
%\usepackage{geometry}                		% See geometry.pdf to learn the layout options. There are lots.
%\geometry{a4paper}                   		% ... or a4paper or a5paper or ... 
%\geometry{landscape}                		% Activate for for rotated page geometry
%\usepackage[parfill]{parskip}    		% Activate to begin paragraphs with an empty line rather than an indent
\usepackage{graphicx}				% Use pdf, png, jpg, or eps§ with pdflatex; use eps in DVI mode
								% TeX will automatically convert eps --> pdf in pdflatex		
\usepackage{amssymb}
\usepackage{lipsum}
\usepackage{underscore}
\usepackage{longtable}

\title{\vspace{-2em}\sc{Basic Oil} QRDC\vspace{-.5em}}
\author{\scriptsize Jean-Luc B\'echennec -- LS2N\\\scriptsize v1.0 -- September 2018}
\date{}							% Activate to display a given date or no date

\begin{document}

\maketitle


\newcommand\ext[1]{\footnotesize #1$\dagger$}
\newcommand\attrval[1]{\footnotesize #1}
\newcommand\defattrval[1]{\footnotesize \hspace{-.3em}\fbox{#1}}
\newcommand\sect[1]{\vspace{-1.5em}\section{\hrulefill\\#1}\vspace{-1.5em}\hrulefill}
\newcommand\code[1]{\texttt{\small #1}}
\newcommand\attr[1]{\vspace{-.8em}\subsection{#1}\vspace{-.3em}}
\newcommand\subattr[2]{\vspace{-.3em}\subsection{{\footnotesize #1} $\triangleright$ #2}}
\newcommand\optattr[1]{\vspace{-.8em}\subsection{#1\,$\dagger$}}
\newcommand\schedservice[1]{\vspace{-.8em}\subsection{#1 $\Join$}}
\newcommand\many{\raisebox{.075em}{$\otimes$}}

%\lipsum[2-56]
\thispagestyle{empty}


\setlength\LTleft\parindent
\setlength\LTright\fill
\setlength\LTpre{.5em}
\setlength\LTpost{-.2em}

The OIL standard defines standard attributes but allows implementation-specific attributes. The latter are suffixed by a $\dagger$. Attribute values in a \fbox{\footnotesize box} are the default values. If none of the attribute values are boxed no default value exists.
Attributes suffixed by a $\otimes$ may appear 0, 1 or many times.

\sect{OS object attributes}

\attr{STATUS}

Set the error processing level of services. See the {\sc Osek} QRDC. Possible values are:

\begin{longtable}{lp{5.8cm}}
\attrval{STANDARD} & does not check object ids, caller identity and state, alarm bounds and resources ownership.\\
\attrval{EXTENDED} & checks all possible errors.\\
\end{longtable}

\attr{STARTUPHOOK}

Switch the call to the startup hook in StartOS.

\begin{longtable}{lp{5.8cm}}
\defattrval{FALSE} & the startup hook is not called\\
\attrval{TRUE} & the startup hook is called\\
\end{longtable}

\attr{SHUTDOWNHOOK}

Switch the call to the shutdown hook in ShutdownOS.

\begin{longtable}{lp{5.8cm}}
\defattrval{FALSE} & the shutdown hook is not called\\
\attrval{TRUE} & the shutdown hook is called\\
\end{longtable}

\attr{ERRORHOOK}

Switch the call to the error hook when an error occurs in a service or while processing an alarm.

\begin{longtable}{lp{5.8cm}}
\defattrval{FALSE} & the error hook is not called\\
\attrval{TRUE} & the error hook is called\\
\end{longtable}

\attr{PRETASKHOOK}

Switch the call to the pre task hook when a context switch is about to occur.

\begin{longtable}{lp{5.8cm}}
\defattrval{FALSE} & the pre task hook is not called\\
\attrval{TRUE} & the pre task hook is called\\
\end{longtable}

\attr{POSTTASKHOOK}

Switch the call to the post task hook when a context switch is about to occur.

\begin{longtable}{lp{5.8cm}}
\defattrval{FALSE} & the post task hook is not called\\
\attrval{TRUE} & the post task hook is called\\
\end{longtable}

\attr{USEGETSERVICEID}

Enable the macro allowing to access in the error hook the identifier of the service in which the error occurred.

\begin{longtable}{lp{5.8cm}}
\defattrval{FALSE} & the macro to get the identifier of the service is disabled\\
\attrval{TRUE} & the macro to get the identifier of the service is enabled\\
\end{longtable}

\attr{USEPARAMETERACCESS}

Enable the macros allowing to access in the error hook the parameters of the service in which the error occurred.

\begin{longtable}{lp{5.8cm}}
\defattrval{FALSE} & the macro to get the parameters of the service is disabled\\
\attrval{TRUE} & the macro to get the parameters of the service is enabled\\
\end{longtable}

\attr{USERESSCHEDULER}

Enable the \code{res_scheduler} resource. This resource has a priority higher than the highest priority task and lower than the lowest priority ISR2.

\begin{longtable}{lp{5.8cm}}
\attrval{FALSE} & the \code{res_scheduler} resource is disabled\\
\defattrval{TRUE} & the \code{res_scheduler} resource is enabled\\
\end{longtable}

\optattr{BUILD}

Enable the generation of a build script.

\begin{longtable}{lp{5.8cm}}
\defattrval{FALSE} & no build script is generated\\
\attrval{TRUE} & a build script is generated\\
\end{longtable}

\subattr{TRUE}{APP_SRC \many}

Gives as a string a C source file name where the application code is located.

\subattr{TRUE}{APP_CPPSRC \many}

Gives as a string a C++ source file name where the application code is located.

\subattr{TRUE}{APP_NAME}

The output binary executable file name.

\subattr{TRUE}{TRAMPOLINE_BASE_PATH}

The path to the trampoline directory.

\subattr{TRUE}{CFLAGS \many}

Gives as a string the flags passed to the C compiler

\subattr{TRUE}{CPPFLAGS \many}

Gives as a string the flags passed to the C++ compiler

\subattr{TRUE}{COMMONFLAGS \many}

Gives as a string the flags passed to both the C and the C++ compiler

\subattr{TRUE}{ASFLAGS \many}

Gives as a string the flags passed to the assembler.

\subattr{TRUE}{LDFLAGS \many}

Gives as a string the flags passed to the linker.

\sect{APPMODE objects attribute}

\attr{DEFAULT}

If $>1$ APPMODE are defined, exactly one of the APPMODE shall have its DEFAULT attribute set to TRUE

\begin{longtable}{lp{5.8cm}}
\defattrval{FALSE} & the APPMODE is not the default one\\
\attrval{TRUE} & the APPMODE is the default one\\
\end{longtable}

\sect{ISR objects attributes}

\attr{CATEGORY}

Set the category of the ISR.

\begin{longtable}{lp{5.8cm}}
\attrval{1} & the ISR is an ISR 1\\
\attrval{2} & the ISR is an ISR 2\\
\end{longtable}


\attr{PRIORITY}

Specifies the priority of the ISR. PRIORITY range from 0 (the lowest) to $2^{32}-1$ (the highest). Even a priority lower than the highest priority task is given, the OIL compiler compute an actual priority greater than the highest priority task.

\attr{SOURCE}

Specifies the hardware interrupt source. Possible values depend on the hardware platform.

\attr{RESOURCE \many}

Gives the resources used by the ISR, usable only for ISR 2.

\attr{MESSAGE \many}

Gives the messages used by the ISR, usable only for ISR 2.

\sect{TASK objects attributes}

\attr{AUTOSTART}

Specifies if a task is in the \code{READY} state or not when the OS is started.

\begin{longtable}{lp{5.8cm}}
\attrval{FALSE} & the task is in the \code{SUSPENDED} state\\
\attrval{TRUE} & the task is in the \code{READY} state\\
\end{longtable}

\subattr{TRUE}{APPMODE \many}

$\geq 1$ APPMODE sub-attribute shall be set. The task is AUTOSTART if the OS is started in one of these APPMODE.


\attr{PRIORITY}

Specifies the priority of the task. PRIORITY range from 0 (the lowest) to $2^{32}-1$ (the highest).

\attr{ACTIVATION}

Specifies the number of jobs that can be recorded. ACTIVATION ranges from 1 to $2^{32}-1$.

\attr{SCHEDULE}

Specifies if the task is preemptable or not.

\begin{longtable}{lp{5.8cm}}
\attrval{FULL} & the task is preemptable\\
\attrval{NON} & the task is not preemptable\\
\end{longtable}

\attr{EVENT \many}

Gives the events used by the task. A task having no EVENT attribute is a basic task. A task having $\geq 1$ EVENT attribut is an extended task.

\attr{RESOURCE \many}

Gives the resources used by the task.

\attr{MESSAGE \many}

Gives the messages used by the task.

\sect{COUNTER objects attributes}

\attr{MINCYCLE}

Set the minimum number of ticks which separate two alarm expirations. Default to 1.

\attr{TICKSPERBASE}

Set the prescaler of the counter. Default to 1.

\attr{MAXALLOWEDVALUE}

Set the maximum value of the counter. Default to 32767.

\sect{ALARM objects attributes}

\attr{COUNTER}

Set the counter used by the alarm.

\attr{AUTOSTART}

Specifies if an alarm is started or not when the OS is started.

\begin{longtable}{lp{5.8cm}}
\attrval{FALSE} & the alarm is not started\\
\attrval{TRUE} & the alarm is started\\
\end{longtable}

\subattr{TRUE}{APPMODE \many}

$\geq 1$ APPMODE sub-attribute shall be set. The alarm is AUTOSTART if the OS is started in one of these APPMODE.

\attr{ACTION}

Specifies which action to do when the alarm expire.

\begin{longtable}{lp{5.8cm}}
\attrval{ACTIVATETASK} & activate a task. \\
\attrval{SETEVENT} & set an event to a task\\
\attrval{ALARMCALLBACK} & call a function\\
\end{longtable}

\subattr{ACTIVATETASK}{TASK}

Specifies which task to activate.

\subattr{SETEVENT}{TASK}

Specifies which task is the target of the event.

\subattr{SETEVENT}{EVENT}

Specifies the event.

\subattr{ALARMCALLBACK}{\\ALARMCALLBACKNAME}

Specifies a string with the name of the function to call.

%\newpage
\sect{EVENT objects attribute}

\attr{MASK}

Set the event mask of the event.

\begin{longtable}{lp{5.8cm}}
\attrval{AUTO} & the actual value is computed by the OIL compiler\\
\attrval{[0..$2^{32}-1$]} & the actual value is set to this value\\
\end{longtable}

\sect{RESOURCE objects attribute}

\attr{RESOURCEPROPERTY}

Set the kind of resource.

\begin{longtable}{lp{5.8cm}}
\attrval{STANDARD} & the resource is a standard one te be used with \code{GetResource} and \code{ReleaseResource}\\
\attrval{INTERNAL} & the resource is an internal one\\
\end{longtable}



\end{document}  