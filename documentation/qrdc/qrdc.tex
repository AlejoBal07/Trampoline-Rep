\documentclass[10pt,notumble]{leaflet}   	% use "amsart" instead of "article" for AMSLaTeX format
%\usepackage{geometry}                		% See geometry.pdf to learn the layout options. There are lots.
%\geometry{a4paper}                   		% ... or a4paper or a5paper or ... 
%\geometry{landscape}                		% Activate for for rotated page geometry
%\usepackage[parfill]{parskip}    		% Activate to begin paragraphs with an empty line rather than an indent
\usepackage{graphicx}				% Use pdf, png, jpg, or eps§ with pdflatex; use eps in DVI mode
								% TeX will automatically convert eps --> pdf in pdflatex		
\usepackage{amssymb}
\usepackage{lipsum}
\usepackage{underscore}
\usepackage{longtable}

\title{\sc{Osek} QRDC\vspace{-.5em}}
\author{\scriptsize Jean-Luc B\'echennec -- LS2N\\\scriptsize v1.0 -- September 2018}
\date{}							% Activate to display a given date or no date

\begin{document}

\maketitle


\newcommand\ext[1]{\footnotesize #1$\dagger$}
\newcommand\std[1]{\footnotesize #1}
\newcommand\sect[1]{\vspace{-1em}\section{\hrulefill\\#1}\vspace{-1.5em}\hrulefill}
\newcommand\code[1]{\texttt{\small #1}}
\newcommand\service[1]{\vspace{-.8em}\subsection{#1}}
\newcommand\schedservice[1]{\vspace{-.8em}\subsection{#1 $\Join$}}

%\lipsum[2-56]
\thispagestyle{empty}


\setlength\LTleft\parindent
\setlength\LTright\fill
\setlength\LTpre{.5em}
\setlength\LTpost{-.2em}

\vspace{-1em}

{\Large\bf Data types}

\begin{longtable}{lp{5.5cm}}
\code{StatusType} & error code returned by a service \\
\code{AppModeType} & an application mode \\
\code{TaskType} & identifier of a task \\
\code{TaskStateType} & state of a task (\code{SUSPENDED}, \code{READY}, \code{RUNNING} or \code{WAITING}) \\
\code{AlarmType} & identifier of an alarm\\
\code{AlarmBaseType} & counter attributes\\
\code{TickType} & number of ticks\\
\code{EventMaskType} & a set of events\\
\code{ResourceType} & identifier of a resource\\
\code{MessageType} & identifier of a message\\
\end{longtable}

\vspace{.5em}

{\Large\bf Services}

Each service returns an error code except \code{GetActiveApplicationMode}. If the OS has been compiled in {\sc Extended} configuration additional error codes may be returned and are suffixed by a $\dagger$. Services suffixed by a $\Join$ lead to a rescheduling.


\sect{Operating system}

\service{StartOS(\emph{app_mode})}

Start the operating system in application mode \emph{app_mode}. Does not return.

\service{ShutdownOS(\emph{error})}

Shutdown the operating system with error code \emph{error}. Does not return.

\service{GetActiveApplicationMode()}

Returns the application mode used to start the operating system.

\sect{Tasks}

%\subsection{}

%-----------------------------------------------------------------------

\schedservice{ActivateTask(\emph{task_id})}

Activate task \emph{task_id}. If task \emph{task_id} has a priority greater than the caller priority, the caller is preempted. Returns:

\begin{longtable}{ll}
\std{E_OK} & Success \\
\std{E_OS_LIMIT} & Too many activation of \emph{task_id} \\
\ext{E_OS_ID} & task \emph{task_id} does not exist\\
\end{longtable}

%-----------------------------------------------------------------------

\schedservice{TerminateTask()}

Terminate the caller. Returns:

\begin{longtable}{ll}
\std{E_OK} & Success \\
\ext{E_OS_RESOURCE} & The caller hold a resource \\
\ext{E_OS_CALLEVEL} & The caller is not a task \\
\end{longtable}

%-----------------------------------------------------------------------

\schedservice{ChainTask(\emph{task_id})}

Terminate the caller and activate \emph{task_id}. Returns:

\begin{longtable}{ll}
\std{E_OK} & Success \\
\std{E_OS_LIMIT} & Too many activation of \emph{task_id} \\
\ext{E_OS_ID} & task \emph{task_id} does not exist\\
\ext{E_OS_RESOURCE} & The caller hold a resource \\
\ext{E_OS_CALLEVEL} & The caller is not a task \\
\end{longtable}

%-----------------------------------------------------------------------

\schedservice{Schedule()}

Call the scheduler. Returns:

\begin{longtable}{ll}
\std{E_OK} & Success \\
\ext{E_OS_RESOURCE} & The caller hold a resource \\
\ext{E_OS_CALLEVEL} & The caller is not a task \\
\end{longtable}

%-----------------------------------------------------------------------

\service{GetTaskID(\emph{task_id_ref})}

Get the task identifier of the task which is currently running. \emph{task_id_ref} is a \underline{\smash{pointer}} to a \code{TaskType} variable where the task identifier of the running task is written. Returns:

\begin{longtable}{ll}
\std{E_OK} & Success \\
\end{longtable}


%-----------------------------------------------------------------------

\service{GetTaskState(\emph{task_id}, \emph{task_state_ref})}

Get the task state of task \emph{task_id}. \emph{task_state_ref} is a \underline{\smash{pointer}} to a \code{TaskState} variable where the  state is written. Returns:

\begin{longtable}{ll}
\std{E_OK} & Success \\
\ext{E_OS_ID} & task \emph{task_id} does not exist\\
\end{longtable}

%-----------------------------------------------------------------------
%-----------------------------------------------------------------------

\sect{Alarms}

%-----------------------------------------------------------------------

\service{GetAlarm(\emph{alarm_id}, \emph{tick_ref})}

Get the remaining tick count of alarm \emph{alarm_id} before the alarm reaches the date. \emph{tick_ref} is a \underline{\smash{pointer}} to a \code{TickType} variable where the remaining tick count  is written. Returns:

\begin{longtable}{ll}
\std{E_OK} & Success \\
\std{E_OS_NOFUNC} & alarm \emph{alarm_id} is not started\\
\ext{E_OS_ID} & alarm \emph{alarm_id} does not exist\\
\end{longtable}

\service{GetAlarmBase(\emph{alarm_id}, \emph{info_ref})}

Get the information about the underlying counter of alarm \emph{alarm_id}. \emph{info_ref} is a \underline{\smash{pointer}} to a \code{AlarmBaseType} variable where the information is written. A \code{AlarmBaseType} is a \code{struct} with 3 fields: \code{maxallowedvalue}, \code{ticksperbase} and \code{mincycle}. Returns:

\begin{longtable}{ll}
\std{E_OK} & Success \\
\ext{E_OS_ID} & alarm \emph{alarm_id} does not exist\\
\end{longtable}

\service{SetRelAlarm(\emph{alarm_id}, \emph{offset}, \emph{cycle})}

Start alarm \emph{alarm_id}. After \emph{offset} ticks the alarm expire and its action is executed. \emph{offset} shall be $>0$. If \emph{cycle} is $>0$ the alarm is restarted and expire every \emph{cycle} ticks. Both \emph{offset} and \emph{cycle} shall $\in [$\code{MINCYCLE}, \code{MAXALLOWEDVALUE}$]$. Returns:

\begin{longtable}{ll}
\std{E_OK} & Success \\
\std{E_OS_NOFUNC} & alarm \emph{alarm_id} is already started\\
\ext{E_OS_ID} & alarm \emph{alarm_id} does not exist\\
\ext{E_OS_VALUE} & \emph{offset} and/or \emph{cycle} out of bounds\\
\end{longtable}

\service{SetAbsAlarm(\emph{alarm_id}, \emph{date}, \emph{cycle})}

Start alarm \emph{alarm_id}. At next counter  \emph{date} the alarm expire and its action is executed. If \emph{cycle} is $>0$ the alarm is restarted and expire every \emph{cycle} ticks. \emph{date} shall be $\leq$ \code{MAXALLOWEDVALUE}. \emph{offset} shall $\in [$\code{MINCYCLE}, \code{MAXALLOWEDVALUE}$]$. Returns:

\begin{longtable}{ll}
\std{E_OK} & Success \\
\std{E_OS_NOFUNC} & alarm \emph{alarm_id} is already started\\
\ext{E_OS_ID} & alarm \emph{alarm_id} does not exist\\
\ext{E_OS_VALUE} & \emph{date} and/or \emph{cycle} out of bounds\\
\end{longtable}

\service{CancelAlarm(\emph{alarm_id})}

Stop alarm \emph{alarm_id}. Returns:

\begin{longtable}{ll}
\std{E_OK} & Success \\
\std{E_OS_NOFUNC} & alarm \emph{alarm_id} is not started\\
\ext{E_OS_ID} & alarm \emph{alarm_id} does not exist\\
\end{longtable}

\sect{Events}

\schedservice{SetEvent(\emph{task_id}, \emph{event_mask})}

Set event(s) \emph{event_mask} to task \emph{task_id}. If task \emph{task_id} was waiting for one of the events of  \emph{event_mask} and it has a higher priority than the caller, the caller is preempted. Returns:

\begin{longtable}{ll}
\std{E_OK} & Success \\
\ext{E_OS_ID} & task \emph{task_id} does not exist\\
\ext{E_OS_ACCESS} & task \emph{task_id} is not an extended task\\
\ext{E_OS_STATE} & task \emph{task_id} is in \code{SUSPENDED} state\\
\end{longtable}

\service{ClearEvent(\emph{event_mask})}

Clear the event(s) of the caller according to events set in \emph{event_mask}. Returns:

\begin{longtable}{ll}
\std{E_OK} & Success \\
\ext{E_OS_ACCESS} & The caller is not an extended task\\
\ext{E_OS_CALLEVEL} & The caller is not a task \\
\end{longtable}

\service{GetEvent(\emph{task_id}, \emph{event_mask_ref})}

Get a copy of the event mask of task \emph{task_id}. \emph{event_mask_ref} is a \underline{\smash{pointer}} to an \code{EventMaskType} variable where the copy is written. Returns

\begin{longtable}{ll}
\std{E_OK} & Success \\
\ext{E_OS_ID} & task \emph{task_id} does not exist\\
\ext{E_OS_ACCESS} & task \emph{task_id} is not an extended task\\
\ext{E_OS_STATE} & task \emph{task_id} is in \code{SUSPENDED} state\\
\end{longtable}

\schedservice{WaitEvent(\emph{event_mask})}

If the none of the events in \emph{event_mask} is set in the event mask of the caller, the caller is put in the \code{WAITING} state. Returns:

\begin{longtable}{ll}
\std{E_OK} & Success \\
\ext{E_OS_ACCESS} & The caller isn't an extended task\\
\ext{E_OS_RESOURCE} & The caller hold a resource \\
\ext{E_OS_CALLEVEL} & The caller is not a task \\
\end{longtable}

\sect{Resources}

\service{GetResource(\emph{rez_id})}

Get resource \emph{rez_id}. The priority of the caller is raised to the priority of the resource if higher. Returns:

\begin{longtable}{lp{5.5cm}}
\std{E_OK} & Success \\
\ext{E_OS_ID} & resource \emph{rez_id} does not exist\\
\ext{E_OS_ACCESS} & The caller try to get a resource already held\\
\end{longtable}

\schedservice{ReleaseResource(\emph{rez_id})}

Release resource \emph{rez_id}. The priority of the caller returns to the priority it had before. Returns:

\begin{longtable}{lp{5.5cm}}
\std{E_OK} & Success \\
\ext{E_OS_ID} & resource \emph{rez_id} does not exist\\
\ext{E_OS_NOFUNC} & The caller try to release a resource it does not hold\\
\ext{E_OS_ACCESS} & The caller try to release a resource with a priority lower than the caller one\\
\end{longtable}

\sect{Messages}

\schedservice{SendMessage(\emph{mess_id}, \emph{data_ref})}

Send message \emph{mess_id}. \emph{data_ref} is a \underline{\smash{pointer}} to a variable containing the data to send. Returns:

\begin{longtable}{lp{6cm}}
\std{E_OK} & Success \\
\ext{E_COM_ID} & message \emph{mess_id} does not exist or has the wrong type\\
\end{longtable}

\schedservice{SendZeroMessage(\emph{mess_id})}

Send signalization message \emph{mess_id}. Returns:

\begin{longtable}{lp{6cm}}
\std{E_OK} & Success \\
\ext{E_COM_ID} & message \emph{mess_id} does not exist or has the wrong type\\
\end{longtable}

\service{ReceiveMessage(\emph{mess_id}, \emph{data_ref})}

Receive message \emph{mess_id}. \emph{data_ref} is a \underline{\smash{pointer}} to a variable where the data are copied.

\begin{longtable}{lp{5.25cm}}
\std{E_OK} & Success \\
\ext{E_COM_ID} & message \emph{mess_id} does not exist or has the wrong type\\
\std{E_COM_NOMSG} & message \emph{mess_id} is queued and the queue is empty\\
\std{E_COM_LIMIT} & message \emph{mess_id} is queued and the queue has overflown\\
\end{longtable}

\service{GetMessageStatus(\emph{mess_id})}

Returns the status of a message:

\begin{longtable}{lp{5.25cm}}
\ext{E_COM_ID} & message \emph{mess_id} does not exist\\
\std{E_COM_NOMSG} & message \emph{mess_id} is queued and the queue is empty\\
\std{E_COM_LIMIT} & message \emph{mess_id} is queued and the queue has overflown\\
\std{E_OK} & None of the above \\
\end{longtable}

\sect{Interrupts}

\service{DisableAllInterrupt()}

Disable all the interrupt sources. Cannot be nested.

\service{EnableAllInterrupt()}

Enable all the interrupt sources. Cannot be nested.

\service{SuspendAllInterrupt()}

Suspend all the interrupt sources. Can be nested.

\service{ResumeAllInterrupt()}

Resume all the interrupt sources. Can be nested.

\service{SuspendOSInterrupt()}

Suspend the interrupt sources of ISR2. Can be nested.

\service{ResumeOSInterrupt()}

Resume the interrupt sources of ISR2. Can be nested.

\end{document}  