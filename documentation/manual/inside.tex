%!TEX root = ./main.tex

\chapter{Kernel Implementation}

\section{The \var{tpl_kern} structure}

The \var{tpl_kern} structure gathers informations about the \RUNNING\ process and flags to notify if a context switch and/or a context save are needed. It eases the access to these informations when programming in assembly language. The \var{tpl_kern} structure is an instance of the \ctype{tpl_kern_state} type:

\begin{lstlisting}[language=C]
typedef struct 
{
  P2CONST(tpl_proc_static, TYPEDEF, OS_CONST) s_old;
  P2CONST(tpl_proc_static, TYPEDEF, OS_CONST) s_running;
  P2VAR(tpl_proc, TYPEDEF, OS_VAR)            old;
  P2VAR(tpl_proc, TYPEDEF, OS_VAR)            running;
  VAR(int, TYPEDEF)                           running_id;
  VAR(u8, TYPEDEF)                            need_switch;
} tpl_kern_state;
\end{lstlisting}

\section{Ready list implementation}

The implementation of the ready list makes it possible to reconcile relative simplicity with good performance regardless of the number of processes\footnote{The term process here refers to a task or an SRI2}. The ready list is implemented by an array indexed by priority. Each element of this array is a FIFO that stores the process identifier. An activated process is stored at the tail of the FIFO. A pre-empted process is stored at the head of the FIFO. Furthermore, in order to quickly find the non-empty FIFO corresponding to the highest priority, a binary heap is used to store the indexes (i.e. priority) of the non-empty FIFOs, the highest index being of course at the root of the heap.
 
FIFO sizes are determined during the OIL compilation. Once the priorities assigned to processes and resources are determined, the FIFO size corresponding to a priority is the sum of the activations of each task, the number of resources and the number of ISR2 for this priority. Priority level 0 is only occupied by the task \emph{idle}.

Let's take for example an application composed of 4 tasks and 2 resources, declared in OIL file as shown below.

\begin{lstlisting}[language=OIL]
RESOURCE r1 { RESOURCEPROPERTY = STANDARD; };
RESOURCE r2 { RESOURCEPROPERTY = STANDARD; };
TASK t1 { PRIORITY = 2; ACTIVATION = 2; RESOURCE = r1; };
TASK t2 { PRIORITY = 3; ACTIVATION = 1; };
TASK t3 { PRIORITY = 1; ACTIVATION = 3; RESOURCE = r2; };
TASK t4 { PRIORITY = 1; ACTIVATION = 2; RESOURCE = r1; RESOURCE = r2; };
\end{lstlisting} 

The missing attributes for all tasks are \oilattr{SCHEDULE = FULL} and \oilattr{AUTOSTART = FALSE}. 

The OIL compiler calculates the priority of the resource \oilval{r1}. As it is likely to be taken by the tasks \oilval{t1} and \oilval{t4}, its calculated priority is 3 (1 higher than the maximum priorities of \oilval{t1} and \oilval{t4}). Therefore the priority of \oilval{t2} is increased by 1 to allow it to pre-empt \oilval{t1} or \oilval{t4} when it holds the resource. The same is done for \oilval{r2}, which leads to give it the priority of 2. Consequently, the priorities of \oilval{t1}, \oilval{t2} and \oilval{r1} are increased by 1 to make room for \oilval{r2}. The occupancy of the priority levels and the corresponding size of the FIFO is therefore as shown in the table \ref{tab:example-fifo}:

% Requires the booktabs if the memoir class is not being used
\begin{table}[htbp]
   \centering
   %\topcaption{Table captions are better up top} % requires the topcapt package
   \begin{tabular}{|llc|} % Column formatting, @{} suppresses leading/trailing space
      \hline
      Priority level    & Content & FIFO size\\
      \hline
      0 & \emph{idle} & 1\\
      1 & \oilval{t3}, \oilval{t4} & 5\\
      2 & \oilval{r2} & 1\\
      3 & \oilval{t1} & 2\\
      4 & \oilval{r1} & 1\\
      5 & \oilval{t2} & 1\\
      \hline
   \end{tabular}
   \caption{Priority levels and FIFO size for the example.}
   \label{tab:example-fifo}
\end{table}% Requires the booktabs if the memoir class is not being used

The data structure for this example is given in figure \ref{fig:array-fifo}. 

\begin{figure}[htbp] %  figure placement: here, top, bottom, or page
   \centering
   \begin{tikzpicture}[yscale=-.8,xscale=.8]
   \draw (-.5,2.5) node[anchor=south, rotate=90] {Priority};
   \foreach \index/\size in {0/1,1/5,2/1,3/2,4/1,5/1} {
      \draw (0,\index) node {\index};
      \draw ($(0,\index)+(.5,-.5)$) rectangle ++(1,1);
      \filldraw ($(0,\index)+(1,0)$) circle (.1);
      \draw[->] ($(0,\index)+(1,0)$) -- ++(1,0);
      \foreach \f in {1,...,\size} {
        \draw ($(.7*\f+.3,\index)+(1,-.4)$) rectangle ++(.7,.7);
      }
   }
   \end{tikzpicture}
   \caption{Data structure for the example.}
   \label{fig:array-fifo}
\end{figure}