%!TEX root = ./main.tex

\chapter{Building a Trampoline application}

An application using trampoline RTOS is made of a set of source files (C/C++) and a set of oil files (.oil). An oil file allows to define the properties of the different objects handled by the RTOS (tasks, resources, alarms, ...).

\note{Trampoline is a static RTOS, all objects are defined at compile time}

The build phase requires a first step to read these OIL files and transform them into data structures (.c/.h files) for the compilation step. This first operation is done with the tool \goil, provided with Trampoline (Figure \ref{fig:filesbis}).

\begin{figure}[htbp]
    \centering
	\begin{adjustbox}{width=.7\linewidth,keepaspectratio}
		\begin{tikzpicture}[
    manager/.style={draw,fill=white,drop shadow={opacity=0.25},font=\scriptsize, text width=1.6cm, text centered, minimum height=1.2cm},
    exe/.style={draw, very thick,fill=white,drop shadow={opacity=0.25},font=\scriptsize, text width=1.2cm, text centered, minimum height=1.2cm},
    src/.style={draw,tape,fill=white,drop shadow={opacity=0.25},font=\scriptsize, text width=1.2cm, text centered, minimum height=1.2cm, tape bend top=out and in, tape bend bottom=out and in},
    srclarge/.style={draw,tape,fill=white,drop shadow={opacity=0.25},font=\scriptsize, text width=2.3cm, text centered, minimum height=1.2cm, tape bend top=out and in, tape bend bottom=out and in}
    ]
        %cadre (car ça dépasse)
\draw[line width=0pt, white] (-0.5,-3.3) rectangle (11.5,3);

    %\draw[fill=gray!30,drop shadow] (0,0) rectangle (10.2,4);
    %\node[,color=red!50,font=\tiny] at (1,1.15) {OS generation process};
    %\node[,color=red!50,font=\tiny] at (1,0.85) {Application generation process};
    %\node[src] (sdmlsrc)          at (1,4)  {SDML Description};
    %\node[manager] (sdmlcompiler) at (4,4)  {SDML Compiler};
    \node[src] (oskernel)         at (7,2)  {Kernel C sources};
    \node[srclarge] (osinfra)     at (10,2) {OS infrastructure\\C sources and ASM};
    %\node[src,text width=2.7cm] (template)	at (4,2)  { System configuration \\ templates};
    \node[src,,text width=2.2cm,fill=yellow!30] (oil)		at (1,0)  { OIL application description};
    \node[src,,text width=2.2cm,fill=yellow!30] (appli)            at (1,-2)  {application sources (C)};
    \node[src,text width=2.2cm] (applidesc)        at (7,0)  {static data structures  \\(C sources)};
    \node[manager,fill=orange!20] (goil)         at (4,0)  {OIL Compiler\\ GOIL v2};
    \node[manager,fill=orange!20] (c)            at (10,-0.5)  {C compiler + linker };
    \node[exe] (exe)              at (10,-2.5)  {binary file};
    %\path (sdmlsrc) edge [dashed,->] (sdmlcompiler);
    %\path (sdmlcompiler) edge [dashed,->] (oskernel);
    \path (oskernel) edge [dashed,->] (c);
    \path (osinfra) edge [dashed,->] (c);
    \path (applidesc) edge [dashed,->] (c);
    \path (appli) edge [dashed,->] (c);
    %\path (sdmlcompiler) edge [dashed,->] (template);
    \path (oil) edge [dashed,->] (goil);
    %\path (template) edge [dashed,->] (goil);
    \path (goil) edge [dashed,->] (applidesc);
    \path (c) edge [dashed,->] (exe);
    %\draw [thick,dashed,color=red!50] (-1,1) -- (12,1);
    %\draw [dashed,->] ($ (messages.south) + (2mm,0) $) -- ++(0,-4mm) -- ($ (events.south) + (0,-4mm) $) -- (events.south);
    %\draw [dashed,->] ($ (messages.south) - (2mm,0) $) -- ++(0,-6mm) -- ($ (tasks.south) + (0,-6mm) $) -- (tasks.south);
    %\draw [dashed,->] ($ (alarms.south) + (2mm,0) $) -- ++(0,-2mm) -- ($ (tasks.south) + (-2mm,-2mm) $) -- ($ (tasks.south) + (-2mm,0) $);
    %\path (alarms) edge [dashed,->] (events);
    %\path (events) edge [dashed,->] (scheduler);
    %\path (tasks) edge [dashed,->] (scheduler);
    %\path (interrupts) edge [dashed,->] (scheduler);
    %\path (resources) edge [dashed,->] (scheduler);
\end{tikzpicture}
	\end{adjustbox}
\caption{Trampoline Application: from source to binary}
	\label{fig:filesbis}
\end{figure}

In addition to the generation of static data structures, \goil\ is able to generate other files, such as those for the definition of memory mapping (link script), tools for debugging (see \ref{chap:trace}), or tools for the compilation of the application. This chapter deals with the generation of the application.

\section{Main OIL file}

To build the application, some additional information is needed, defined in the sub-attribute \oilattr{CPU->OS->BUILD}. Example for ARM Cortex-M target:

\lstset{language=OIL}
\begin{lstlisting}
CPU blink {
  OS config { 
    BUILD = TRUE {
      TRAMPOLINE_BASE_PATH = "/opt/trampoline";
      APP_SRC = "blink.c";
      APP_NAME = "blink_exe";
      CFLAGS  = "-O0";
      LDFLAGS = "-Map=blink.map";
      COMPILER = "arm-none-eabi-gcc";
      ASSEMBLER = "arm-none-eabi-as";
      LINKER = "arm-none-eabi-ld";
      COPIER = "arm-none-eabi-objcopy";
      SYSTEM = PYTHON;
    };
  ..
\end{lstlisting}

\note{For each target, at least one example is provided in the tree structure \file{examples/}. This is a good starting point.}

This attribute is specific to Trampoline, it contains several sub-attributes to build the application (cross-compiler, flags, source files, …):
\begin{itemize}
	\item \oilattr{TRAMPOLINE_BASE_PATH}: The path to the root of Trampoline. Required
	\item \oilattr{APP_SRC}, \oilattr{APP_CPPSRC} give respectively the C and C++ source files of the application.
	\item \oilattr{CFLAGS}, \oilattr{CPPFLAGS}, \oilattr{ASFLAGS}, \oilattr{LDFLAGS} define the flags to give to the compiler for respectively C, C++, assembly, linker phase. \oilattr{COMMONFLAGS} gives flags for both c,C++and asm source files.
	\item \oilattr{APP_NAME} define the name of the output binary file (to flash)
	\item \oilattr{COMPILER}, \oilattr{CPPCOMPILER}, \oilattr{ASSEMBLER}, \oilattr{LINKER} and \oilattr{COPIER} are the tools (cross compiler collection) related to the C compiler, C++ compiler, assembler, linker and copier (as the GNU tool objcopy)
	\item \oilattr{SYSTEM} defines the build tool in use. It can be either \oilattr{PYTHON} (default) to generate a set of python scripts, or \oilattr{CMAKE} to use the CMake build tool. See section \ref{sec:buildsystem}.
\end{itemize}

As this description is in OIL language, if a sub-attribute is defined twice, then it will be accumulated. For instance:
\lstset{language=OIL}
\begin{lstlisting}
  APP_SRC = "blink.c";
  APP_SRC = "file2.c";
\end{lstlisting}
is equivalent to:
\begin{lstlisting}
  APP_SRC = "blink.c file2.c";
\end{lstlisting}

\section{Build system}
\label{sec:buildsystem}
2 build system are available for Trampoline, defined in the sub-attribute \oilattr{CPU->OS->BUILD->SYSTEM}:
\begin{itemize}
	\item \oilattr{PYTHON} is a set of python script
	\item \oilattr{CMAKE} is a set of CMake files 
\end{itemize}

\subsection{Python build}
The python system generates 2 files \file{build.py} and \file{make.py}. The script will take into account all the dependancies. For example, modifying an object in the oil file will result in calling goil (and generating again the \file{build.py} file again), before doing the rest of the build step. As a result, \goil\ should be called only once (bootstrap), and then \com{./make.py} will do all the stuff.
A basic run is (from \file{examples/cortex/armv7em/stm32f303/Nucleo-32/blink})
\begin{verbatim}
% goil --target=cortex/armv7em/stm32f303 --templates=../../../../../../goil/templates/ blink.oil
Created 'blink/tpl_dispatch_table.c'.
Created 'blink/tpl_invoque.S'.
Created 'blink/tpl_os.h'.
Created 'blink/tpl_service_ids.h'.
Created 'build.py'.
Created 'make.py'.
Created 'blink/tpl_app_custom_types.h'.
Created 'blink/tpl_app_config.c'.
Created 'blink/tpl_app_config.h'.
Created 'blink/tpl_app_define.h'.
Created 'blink/MemMap.h'.
Created 'blink/Compiler.h'.
Created 'blink/Compiler_Cfg.h'.
Created 'blink/script.ld'.
Created 'blink/AsMemMap.h'.
Created 'blink/tpl_static_info.json'.
Created 'blink/tpl_vectors.c'.
Created 'blink/tpl_primary_irq.S'.
Created 'blink/cmsis_wrapper.h'.
Created 'blink/tpl_external_interrupts.c'.
Created 'blink/tpl_app_interrupts.c'.
Created 'blink/tpl_cortex_definitions.h'.
Created 'blink/stm_structure.c'.
executing plugin gdb_commands.goilTemplate
Created '/home/mik/prog/trampoline/examples/cortex/armv7em/stm32f303/Nucleo-32/blink/build/blink.oil.dep'.
No warning, no error.
\end{verbatim}
Then, we call the python script \com{./make.py}:
\begin{verbatim}
-> % ./make.py 
Nothing to make.
Making "build/os" directory
[  3%] Compiling ../../../../../../os/tpl_os_kernel.c
[  6%] Compiling ../../../../../../os/tpl_os_timeobj_kernel.c
…
[ 96%] Compiling ../../../../../../machines/cortex/armv7em/stm32f303/tpl_trace.c
[100%] Linking blink_exe
		   Generating binary blink_exe.bin from blink_exe	
\end{verbatim}

In most cases, an additional target has been defined to flash the application (see section \ref{sec:additionnalTarget}). For Cortex-M targets, this is \com{./make.py burn}.

\subsection{CMake build system}

The CMake build system generates 2 text based files:
\begin{itemize}
	\item \file{CMakeLists.txt} is the main project file
	\item \file{blink/compiler.cmake} (if project is defined in \file{blink.oil}) defines cross-compiler definitions.
\end{itemize}

The bootstrap using \goil\ is the same as in PYTHON:
\begin{verbatim}
% goil --target=cortex/armv7em/stm32f303 --templates=../../../../../../goil/templates/ blink.oil
Created 'blink/tpl_dispatch_table.c'.
Created 'blink/tpl_invoque.S'.
Created 'blink/tpl_os.h'.
Created 'blink/tpl_service_ids.h'.
Created 'CMakeLists.txt'.
Created 'blink/compiler.cmake'.
Created 'blink/tpl_app_custom_types.h'.
Created 'blink/tpl_app_config.c'.
Created 'blink/tpl_app_config.h'.
Created 'blink/tpl_app_define.h'.
Created 'blink/MemMap.h'.
Created 'blink/Compiler.h'.
Created 'blink/Compiler_Cfg.h'.
Created 'blink/script.ld'.
Created 'blink/AsMemMap.h'.
Created 'blink/tpl_static_info.json'.
Created 'blink/tpl_vectors.c'.
Created 'blink/tpl_primary_irq.S'.
Created 'blink/cmsis_wrapper.h'.
Created 'blink/tpl_external_interrupts.c'.
Created 'blink/tpl_app_interrupts.c'.
Created 'blink/tpl_cortex_definitions.h'.
Created 'blink/stm_structure.c'.
executing plugin gdb_commands.goilTemplate
Created '/home/mik/prog/trampoline/examples/cortex/armv7em/stm32f303/Nucleo-32/blink/build/blink.oil.dep'.
No warning, no error.
\end{verbatim}

Then, the command to generate the binary are the standard way, except a special attention to define the correct cross-compiler when calling \tool{cmake}, using the generated cross-compiler access: \com{cmake -D CMAKE_TOOLCHAIN_FILE=../blink/compiler.cmake ..}

\note{These commands are defined in comments in the head of \file{CMakeLists.txt}.}

\begin{verbatim}
% cd build    
% cmake -D CMAKE_TOOLCHAIN_FILE=../blink/compiler.cmake ..
-- The C compiler identification is GNU 9.3.1
-- The CXX compiler identification is GNU 9.4.0
-- Detecting C compiler ABI info
-- Detecting C compiler ABI info - done
-- Check for working C compiler: /opt/gcc-arm/bin/arm-none-eabi-gcc - skipped
-- Detecting C compile features
-- Detecting C compile features - done
-- Detecting CXX compiler ABI info
-- Detecting CXX compiler ABI info - done
-- Check for working CXX compiler: /usr/bin/g++ - skipped
-- Detecting CXX compile features
-- Detecting CXX compile features - done
-- The ASM compiler identification is GNU
-- Found assembler: /opt/gcc-arm/bin/arm-none-eabi-gcc
-- Configuring done
-- Generating done
-- Build files have been written to: /home/mik/prog/trampoline/examples/cortex/armv7em/stm32f303/Nucleo-32/blink/build
-> % make
Scanning dependencies of target blink_exe
[  3%] Building C object CMakeFiles/blink_exe.dir/blink.c.obj
[  6%] Building C object CMakeFiles/blink_exe.dir/home/mik/
...
[ 96%] Linking C executable blink_exe
[100%] Built target blink_exe
\end{verbatim}

In most cases, an additional target has been defined to flash the application (see section \ref{sec:additionnalTarget}). For Cortex-M targets, this is \com{make burn}.

\warning{If the oil file is updated, the build system will call \goil. \goil\ may update \file{CMakeLists.txt}, but this modification will not be taken into account during this pass. In that case, the command \com{make} should be run twice to get a correct behavior.}

The interest of using CMake is that several IDEs are based on this system for the project management (Qt Creator{\footnote{https://www.qt.io/product/development-tools}}, VS Code{footnote{https://code.visualstudio.com/}}, ...). Therefore, we can take advantage of the features of these editors (code completion, navigation between files, ...)

A special case is for VS Code editor. With an additional sub-atribute, a \file{.vscode/} directory is generated (if it does not exist yet), so that the debugger is well configured. In that way, you can directly compile and debug from the editor.
It runs for some Cortex-M based target at this date (STM32).
\lstset{language=OIL}
\begin{lstlisting}
 SYSTEM = CMAKE { VSCODE=TRUE; };
\end{lstlisting}

\section{Goil related features}

Many parts of the build system are generated from \goil\ templates. This section introduces these features.

\subsection{Compilation flags}

Some flags may be defined globally for a specific target. It can be retrieved in the \goil\ templates, in the \file{config/} subdirectories. The file is named \file{buildOptions.oil}. For instance, on ARM Cortex-M4 (armv7em ISA), the path is \file{goil/templates/config/cortex/armv7em/buildOptions.oil}.

Some other flags can be added directly in the main \file{config.oil} file in the templates.

\subsection{Additionnal files}

Depending on the target, additional files may need to be compiled. For example, for the \posix\ target, the C files located in the \file{machines/posix} directory must be compiled to enable Trampoline and the application to run in a Unix process. The list of required files is given in one or many OIL objects located along the path in the \file{config} subdirectory of the template directory.

For example, for the target \posix, the list of additional files is as follows:

\begin{lstlisting}[language=OIL]
PLATFORM_FILES posix_port {
    PATH = "posix";
    CFILE = "tpl_machine_posix.c";
    CFILE = "tpl_viper_interface.c";
    CFILE = "tpl_posix_autosar.c";
    CFILE = "tpl_posix_irq.c";
    CFILE = "tpl_posix_context.c";
    CFILE = "tpl_posixvp_irq_gen.c";
};

PLATFORM_FILES posix_port_trace {
    IF = TRACE;
    PATH = "posix";
    CFILE = "tpl_trace.c";
};
\end{lstlisting}

The first object, \lstinline{posix_port}, lists the files that are always required, and the second, \lstinline{posix_port_trace}, lists the file \file{tpl_trace.c} that is required if trace is enabled.

A \oilobj{PLATFORM_FILES} object has the following attributes:

\begin{description}
\item[\texttt{PATH}] is mandatory. It indicates in which directory, relative to the \file{machines} directory, the files are located;
\item[\texttt{CFILE}] gives a C file to be compiled. Several \oilattr{CFILE} attributes may be present;
\item[\texttt{CPPFILE}] gives a C++ file to be compiled. Several \oilattr{CPPFILE} attributes may be present;
\item[\texttt{ASFILE}] gives a assembly language file to be compiled. Several \oilattr{ASFILE} attributes may be present;
\item[\texttt{IF}] refers to a Boolean attribute of the OS object. If this attribute is \oilval{TRUE}, the files are compiled, otherwise they are not.
\end{description}

For example, the following statement indicates that the file \file{machines/tpl_trace.c} should be compiled if the \oilattr{TRACE} attribute of the \oilattr{OS} object is \oilval{TRUE}.

\begin{lstlisting}[language=OIL]
PLATFORM_FILES posix_port_trace {
    IF = TRACE;
    PATH = "posix";
    CFILE = "tpl_trace.c";
};
\end{lstlisting}


\subsection{Libraries}
\label{sec:libraries}

Libraries are software modules that can potentially be used on several targets. These include device drivers common to several targets from the same manufacturer, as well as software functionalities that are independent of all targets.
Libraries are grouped together in a directory hierarchy whose root is the \file{libraries} directory which is located at the root directory of Trampoline.

\subsubsection{How to package a library}

Assuming that the directory where Trampoline is installed is named \file{trampoline}, the library source code is located in the \file{libraries} subdirectory. The directory hierarchy reflects the type of library. For example, the CAN library is located in \file{libraries/net/can} and the dummy CAN controller library is located in \file{libraries/net/can/demo_driver}.

In the template directory, configuration files describe where the source code of a library is located relative to the \file{libraries} directory. Thus, the description of the CAN library is as follows:

\begin{lstlisting}[language=OIL]
CPU net_can_files {
  LIBRARY can {
    PATH = "net/can";
    CFILE = "tpl_can_core.c";
  };
};
\end{lstlisting} 

And the description of the demo_driver is as follow:

\begin{lstlisting}[language=OIL]
CPU net_can_demo_driver_files {
  LIBRARY net_can_demo_driver {
    PATH = "net/can/demo_driver";
    CHEADER = "tpl_can_demo_driver.h";
    CFILE = "tpl_can_demo_driver.c";
  };
};
\end{lstlisting} 

Not all libraries are usable on all platforms and it is necessary to specify which libraries are available for a given platform. This is done in the target's \file{config.oil} file. So, for the posix target, the \file{config/posix/config.oil} file describes the available libraries as follows:

\begin{lstlisting}[language=OIL]
LIBRARY net_can_demo_driver {
  GLOBAL = TRUE;
  PATH = "net/can/demo_driver";
};

LIBRARY can {
  GLOBAL = TRUE;
  NEEDS = net_can_demo_driver;
  PATH = "net/can";
};
\end{lstlisting} 

For both, \oilattr{GLOBAL = TRUE;} indicates a library in the \file{libraries} hierarchy\footnote{Specific libraries may also reside in the target hierarchy. In this case, \oilattr{GLOBAL} is set to false, which is the default value.}. \oilattr{PATH} specifies the subdirectory where the \file{config.oil} file describing the library is located (see above). \oilattr{NEEDS} indicates a dependency relationship between libraries. Here, the \oilobj{can} library requires the \oilobj{net_can_demo_driver} library to function.

Finally, a \oilattr{LIBRARY} attribute of type \ctype{ENUM} is added to the OS object to list the libraries that the user can reference. As it makes no sense to reference the \oilobj{net_can_demo_driver} library directly, only the \oilobj{can} library is listed in the \ctype{ENUM}. The possible ENUM values must match the object names \oilattr{LIBRARY}:

\begin{lstlisting}[language=OIL]
ENUM [
  can
] LIBRARY[];
\end{lstlisting} 

\subsubsection{Other attributes of the \oilattr{LIBRARY} object}

These attributes can appear as many times as required.

\begin{description}
  \item[\oilattren{CHEADER}] gives the name of a header file in C language;
  \item[\oilattren{CPPHEADER}] allows you to name a header file in C++ language;
  \item[\oilattren{CFILE}] to give the name of a source file in C language;
  \item[\oilattren{CPPFILE}] to name a source file in C++ language;
  \item[\oilattren{ASFILE}]  to name a source file in assembly language.
 \end{description}

\subsubsection{How to use a library}

Libraries are selected via a series of \oilattr{LIBRARY = <name>;} in the OS object. During build script generation, \goil, resolves library dependencies, inserts the necessary source files into the files to be compiled and inserts the paths into the include path list.


\subsection{Additionnal build target}
\label{sec:additionnalTarget}
TODO
